%% Author: Daniel Kaplan
%% Subject: Model design, threshold regression
%% Date: 2007


Consider the following situation. In order to encourage schools to perform
well, a school district hires an external evaluator to give a rating to
each school in the district.  The rating is a single number based on
the quality of the teachers, absenteeism among the teachers, the
amount and quality of homeworks the the teachers assign, and so on.  

To reward those schools that do well, the district gives a moderate
salary bonus to each teacher and a fairly large budget increase to the
school itself.

The next year, the school district publishes data showing that the
students in schools that received the budget increases had done much
better on standarized test scores than the schools that hadn't gotten
the increases.  The school district argues that this means that
increasing budgets overall will improve performance on standardized tests.

The teacher's union supports the idea of higher budgets, but objects
to the rating system,  saying that it is
meaningless and that teacher pay should not be linked to it.  The Taxpayers
League argues that there is no real evidence that higher spending
produces better results.  They interpret the school district's data as
indicating only that the higher ranked schools are better and, of
course, better schools give better results.  Those schools were better
before they won the ratings-based budget increase.

This is a serious problem.  Because of the way the school district
collected its data, being a high-rated school is confounded with
getting a higher budget.  

A modeling technique for dealing with situations like this is called
\newword{threshold regression}.  Threshold regression 
models student test scores at each school as a function of the school
rating, but includes another variable that indicates whether the school
got a budget increase.  The budget increase variable is almost the
same thing as the school rating: because of the way the school
district awarded the increases, it is a threshold transformation of
the school rating.

% see Language/figures.r threshold.example(noise=60)
\centerline{\includegraphics[width=3.5in]{../Figures/school-threshold-example.png}}

The graph shows some data (plotted as circles) 
from a simulation of this situation in which the budget increase 
had a genuine impact of 50 points in the standardized test.  The
solid line shows the model of test score as a function of school
rating, with only the main effect. This model corresponds to the claim
that the threshold has no effect.  The solid dots are the model
values from another model, with rating as a main effect and a
threshold transformation of rating that corresponds to which schools
got the budget increase.

Explain how to interpret the models as indicating the effect of the
budget increase.  In addition to your explanation, make sure to give a
numerical estimate of how big the effect is, according to the model as
indicated in the graph.
\TextEntry

\begin{AnswerText}
The model that includes the threshold transformation term shows a jump
at the threshold.  By the nature of the model, any jump would have to
be exactly at the threshold: that's how we defined the threshold
term.  It's the size of the jump that indicates what's going on.  This
jump is positive for the schools getting the budget increase and
amounts to about 40 points in test scores.
\end{AnswerText}

An important statistical question is whether the data provide good
support for the claim that the threshold makes a difference.  
(Techniques for answering this question are discussed later in the book).  The
answer depends both on the size of the effect, and how much data is
used for constructing the model.  For
the simulation here, it turns out that the threshold model has
successfully detected the effect of the budget increase.

