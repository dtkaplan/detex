%% Author: George Cobb}
%% Subject: Variables, Variable types,Response and explanatory

In studies of employment discrimination, several attributes of
employees are often relevant:
\begin{quotation}
\noindent
age,  sex, race, years of experience, salary, whether promoted, 
whether laid off 
\end{quotation}

For each of the following questions, indicate
which is the response variable and which is the explanatory variable.

\begin{enumerate}
\item Are men paid more than women? 

Response Variable:\\
\matchSelect{age,sex,race,years.experience,salary,promoted,laid.off}{salary}

Explanatory Variable:\\
\matchSelect{age,sex,race,years.experience,salary,promoted,laid.off}{sex}

\begin{AnswerText}
For many people, it can be confusing at first to decide what is the
response variable and what is the explanatory variable.  In this
question, there are two variables: salary and sex.  There are two
kinds of questions you might ask with these variables: ``Can I tell
something about how much a person is paid given their sex?'' and ``Can
I tell what sex a person is, given how much they are paid?''  Those
questions are different.  

In examining possible employment discrimination, the issue is whether
a person's sex influences how much they are paid.  So sex is the
explanatory variable and pay is the response variable.
\end{AnswerText}

\item On average, how much extra salary is a year of experience worth? 

Response Variable:\\
\matchSelect{age,sex,race,years.experience,salary,promoted,laid.off}{salary}

Explanatory Variable:\\ 
\matchSelect{age,sex,race,years.experience,salary,promoted,laid.off}{years.experience}

\begin{AnswerText}
The response variable is the person's salary.  The explanatory
variable is how much experience they have.
\end{AnswerText}

\item Are whites more likely than blacks to be promoted? 

Response Variable:\\
\matchSelect{age,sex,race,years.experience,salary,promoted,laid.off}{promoted}

Explanatory Variable:\\
\matchSelect{age,sex,race,years.experience,salary,promoted,laid.off}{race}

\begin{AnswerText}
The response variable is whether the person was promoted or not.
\end{AnswerText}

\item Are older employees more likely to be laid off than younger ones?  

Response Variable:\\
\matchSelect{age,sex,race,years.experience,salary,promoted,laid.off}{laid.off}

Explanatory Variable:\\ 
\matchSelect{age,sex,race,years.experience,salary,promoted,laid.off}{age}
\end{enumerate}

\begin{AnswerText}
In these exercises, there is just one explanatory variable in each setting.  However,
in real situations you will often be interested in 
examining {\bf multiple} explanatory variables at the same time.  
\end{AnswerText}
\bigskip

[Note: Thanks to George Cobb.]